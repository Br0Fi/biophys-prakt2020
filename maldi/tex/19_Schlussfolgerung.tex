\newpage
\section{Schlussfolgerung}
	% Rückgriff auf Hypothese und drittes Nennen dieser
	% Quellen zitieren, Websiten mit Zugriffsdatum
	% Verweise auf das Laborbuch (sind erlaubt)
	% Tabelle + Bilder mit Beschriftung

  Zusammenfassend lässt sich sagen, dass die anzustellenden Untersuchungen erfolgreich durchgeführt werden konnten.

  So konnte gezeigt werden, dass der Sensitivity-Modus eine deutlich höhere Ionenausbeute, aber dafür geringere Massenauflösung als der High-Resolution-Modus erlaubt.
  Außerdem wurde festgestellt, dass im untersuchten Bereich eine stärkere Verdünnung zu schlechterem Signal-Rausch-Verhältnis führt.
  Eine Abhängigkeit der Massenrichtigkeit und Massengenauigkeit von der Verdünnung konnte nicht eindeutig nachgewiesen werden.
  Es ist auch zu erwähnen, dass aufgrund von Probenkontamination nicht alle aufgenommenen Spektren ausgewertet werden konnten und in einem Falle vermutlich ein Fehler in der Verdünnung aufgetreten ist.

  Eine Bestimmung der Schwellfluenz für das Probensystem war aufgrund von technischen Problemen nicht möglich.

  MALDI-Imaging hat erlaubt, einen unbekannten Maushirn-Coronal-Schnitt einer Position im Maushirn zuzuordnen und anhand von verschiedener Ionen und mittels Referenzbildern unterschiedliche Bereiche im Hirn zu identifizieren.
