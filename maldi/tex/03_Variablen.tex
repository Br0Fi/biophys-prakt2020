% Autor: Simon May
% Datum: 2016-10-13
% Der Befehl \newcommand kann auch benutzt werden, um „Variablen“ zu definieren:

% Nummer laut Praktikumsheft:
%\newcommand*{\varNum}{V07}
% Name laut Praktikumsheft:
\newcommand*{\varName}{MALDI: Lipide und Imaging} %"\\" in hyperref gives warnings and is removed.
% Datum der Durchführung (Format: JJJJ-MM-TT):
\newcommand*{\varDatum}{01.09.2020}
% Autoren des Protokolls:
\newcommand*{\varAutor}{N. Wydra, A. Oster, L. Segger}
\newcommand*{\varNameA}{Norbert Wydra}
\newcommand*{\varNameB}{Alex Oster}
\newcommand*{\varNameC}{Leonhard Segger}
% Nummer der eigenen Gruppe:
\newcommand*{\varGruppe}{Blockpraktikum Biophysik, Gruppe F}
% E-Mail-Adressen der Autoren (kommagetrennt ohne Leerzeichen!):
\newcommand{\varEmail}{l.segger@uni--muenster.de,a\_oste16@uni--muenster.de, n\_wydr01@uni--muenster.de} %TODO wydr oder wydr01?
\newcommand{\varEmailA}{n\_wydr01@uni--muenster.de}
\newcommand{\varEmailB}{a\_oste16@uni--muenster.de}
\newcommand{\varEmailC}{l.segger@uni--muenster.de}
%betreuer Name
\newcommand{\varBetreuer}{\normalsize betreut von Jan Schwenzfeier und Olaf Minte}
% E-Mail-Adresse anzeigen (true/false):
\newcommand*{\varZeigeEmail}{true}
% Kopfzeile anzeigen (true/false):
\newcommand*{\varZeigeKopfzeile}{true}
% Inhaltsverzeichnis anzeigen (true/false):
\newcommand*{\varZeigeInhaltsverzeichnis}{true}
% Literaturverzeichnis anzeigen (true/false):
\newcommand*{\varZeigeLiteraturverzeichnis}{true}
