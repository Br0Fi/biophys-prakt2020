\section{Durchführung}

\subsection{Lipide}
Als Analyt wird ein Gemisch aus Phosphatidylethanolaminen (PE) und Phosphatidylcholinen (PC) verwendet.
Das Gemisch besteht aus \SI{0,38}{mmol/l} PE(36:2), \SI{0,31}{mmol/l} PE(36:0), \SI{0,31}{mmol/l} PE(32:0), \SI{0,35}{mmol/l} PC(34:1) und \SI{0,34}{mmol/l} PC(32:0) gelöst in CHCl$_3$/MeOH (2/1 v/v). %muss man irgendwie schon nennen.
Es wird eine Targetplatte mit der \enquote{dried droplet}-Präparation mit zwei verschiedenen Matrizes (DHB und CHCA) und jeweils vier verschiedenen Konzentrationen des Analyten beschichtet.
Die Verdünnungsreihe besteht aus dem Gemisch verdünnt auf Volumenanteile von $10^{-4}$ ,$10^{-5}$,$10^{-6}$ und $10^{-7}$ in Bezug auf die oben beschriebene Ursprungslösung.
Verdünnt wird auch hier mit CHCl$_3$/MeOH (2/1 v/v).
Auf der Targetplatte werden jeweil drei Spots angelegt, indem zuerst \SI{1}{\micro \liter} Matrixlösung und dann \SI{1}{\micro \liter} Analytlösung augegeben werden.

Für die MALDI wird ein N2-Laser verwendet und eine Kalibrierung anhand von bekannten Peaks in einer Probe von violettem Phosphor durchgeführt.
Eine Bestimmung der Schwellfluenz des Lasers kann aufgrund eines Fehlers am Gerät nicht durchgeführt werden.

Es werden jeweils mehrere Messungen im Sensitivity- und im High-Resolution-Modus durchgeführt (im Folgenden S- und HR-Modus), wobei für jede Messung 30 Spektren aufsummiert werden.
Diese unterscheiden sich insofern, als dass im HR-Modus im Flugzeitspektrometer insgesamt drei Reflektoren verwendet werden, sodass die Ionen in einer W-Form fliegen.
Dies soll die Massenauflösung erhöhen, da die Verteilung der Anfangsenergien der Ionen besonders gut kompensiert wird, verringert aber die Sensitivität, also die Anzahl der gemessenen Ionen.

Die übrigen Messungen zum Vergleich der Analytkonzentrationen werden im Sensitivity-Modus durchgeführt.

Während der Messungen wird der Spot jeweils unter dem Laserstrahl bewegt, damit nicht nur eine Stelle des Spots beschossen wird.

\subsection{Imaging}

MALDI-Imaging wird verwendet, um eine ortsaufgelöste Aufnahme der Lipidverteilung in einer organischen Probe zu erlangen.
Dazu wird ein mit Matrix (\SI{20}{mg/ml} 2,5-DHB in ACN/H$_2$O, 70/30) beschichteter Maushirn-Dünnschnitt verwendet.
Über diesen wird dann ein Gitter abgerastert, bei dem für jeden Gitterpunkt ein Massenspektrum aufgenommen wird.
Dies dauert einige Stunden, weshalb bereits zuvor aufgenommene Daten verwendet werden.
