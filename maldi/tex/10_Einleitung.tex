\section{Einleitung}
	% Hypothese	und deren Ergebnis, wenn Hypothese ist, dass nur Theorie erfüllt, sagen: Erwartung: Theorie aus einführung (mit reflink) erfüllt
	% Ergebnisse, auch Zahlen, mindestens wenn's halbwegs Sinn ergibt
	% Was wurde gemacht
	% manche leute wollen Passiv oder "man", manche nicht


  MALDI (Matrix-unterstützte Laser-Desorption/Ionisation) erlaubt es, Moleküle in die Gasphase zu überführen und zu ionisieren.
  In der MALDI wird dies erreicht, indem Analytmoleküle zusammen mit einer Matrix in einen Kristall eingebaut werden, welcher dann mit einem Laser beschossen wird.
  In der Gasphase können sie dann mittels eines Massenspektrometers (hier eines Flugzeitspektrometer) nach ihrem Masse-zu-Ladung-Verhältnis $m/z$ aufgetrennt und nachgewiesen werden.

  Dieses Verfahren lässt verschiedene Parameter offen, die für Ionenausbeute, geringe Ionenfragmentierung, Signal-Rausch-Verhältnis, Massenauflösung und Massengenauigkeit optimiert werden können.
  Dementsprechend sollen hier die Parameter Analytverdünnung, Art der verwendeten Matrix und Modus des Flugzeitspektrometers verändert werden, um die Abhängigkeit jener Größen zu untersuchen.

  Des Weiteren ist es möglich, räumlich aufgelöste MALDI-Messungen durchzuführen, indem eine dünn geschnittene Probe komplett mit Matrix beschichtet wird und dann rasterförmig MALDI-Spektren für jeden Pixel des Rasters aufgenommen werden.
  Dies kann beispielsweise erlauben, während Operationen festzustellen, welche Bereiche eines Organs von Krebs befallen sind und welche nicht.
  Hierfür muss bekannt sein, wie sich die Häufigkeiten bestimmter Ionen in Tumorzellen von denen in gesunden Zellen unterscheiden.
  Solange dies bekannt ist, ist es jedoch nicht nötig, sich auf Anfärbungen und Erfahrung des/der OP-Leiter:in zu verlassen.

  In diesem Versuch soll ein Coronal-Schnitt eines Mäusehirns verwendet werden, um Strukturen innerhalb der Probe zu identifizieren und den Schnitt innerhalb des Hirns zu lokalisieren.
  Dazu werden Referenzbilder, welche anhand von verschiedenen Färbungen entwickelt wurden, verwendet.

  MALDI ist nicht in der Lage, zu quantifizieren, wie viele Ionen eines bestimmten Typs in der Probe vorhanden waren, da ein Modell, welches den Zusammenhang zwischen Ionenkonzentration in der Probe und Anzahl der gemessenen Ionen sehr komplex wäre und keine einheitliche quantifizierende Theorie zum Mechanismus von Desorption und Ionisation existiert.

  Aufgrund von Matrixeffekten und probenabhängiger schwankender Ionenausbeiten ist es jedoch häufig schwierig, mit der MALDI eine Quantifizierung der Konzentration der gemessenen Ionen in der Probe vorzunehmen.
