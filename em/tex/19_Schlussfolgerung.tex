\newpage
\section{Schlussfolgerung}
	% Rückgriff auf Hypothese und drittes Nennen dieser
	% Quellen zitieren, Websiten mit Zugriffsdatum
	% Verweise auf das Laborbuch (sind erlaubt)
	% Tabelle + Bilder mit Beschriftung

  Zusammenfassend lässt sich sagen, dass die Elektronenmikroskopie auch für biologische Proben ein wichtiges Verfahren  zur Untersuchung kleinster Strukturen ist.
  Dabei ist deutlich geworden, wie groß die Rolle der Probenpräparation ist, da sie bestimmt, ob die zuuntersuchenden Strukturen in vollständig präparierten Probe überhaupt noch vorhanden sind.

  Mittels SEM wurden ganze Insekten mit hohem Detailgrad untersucht und die Probleme der simplen Trocknung an Atmosphärenluft aufgezeigt.
  Außerdem wurde der Einfluss von Astigmatismus visualisiert und anhand des Beispiels eines Mikrochips der Unterschied zwischen BSE- und SE-Aufnahmen deutlich gemacht.

  TEM wurde genutzt, um Astrozyten bei verschiedenen Vergrößerungen zu untersuchen.
  Dabei konnten Nukleus, Zytoplasma, Golgi-Apparat und Mitochondrien sichtbar gemacht werden.

  Außerdem wurde die Struktur von Herzmuskeln untersucht, wobei unterschiedliche funktionelle Einheiten für den Kontraktionsprozess beobachtet werden konnten.
