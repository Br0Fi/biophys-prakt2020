\section{Einleitung}
	% Hypothese	und deren Ergebnis, wenn Hypothese ist, dass nur Theorie erfüllt, sagen: Erwartung: Theorie aus einführung (mit reflink) erfüllt
	% Ergebnisse, auch Zahlen, mindestens wenn's halbwegs Sinn ergibt
	% Was wurde gemacht
	% manche leute wollen Passiv oder "man", manche nicht

  In diesem Versuch soll der Nutzen des Elektronenmikroskops bei der Untersuchtung von kleinsten biologischen Strukturen gezeigt werden.
  Dazu werden nach einer Einführung in die verwendeten Verfahren die Vor- und Nachteile dieser Verfahren anhand verschiedener Aufnahmen deutlich gemacht.
  Des Weiteren wird auf die Notwendigkeit einer komplexen Probenpräparation eingegangen und die Effekte einer schlechten Probenpräparation visualisiert.

  Rasterelektronenmikroskopie wird verwendet, um Trocknungsschäden an einem Insektenauge zu zeigen und die Bedeutung von Astigmatismus abzubilden.
  Hierbei wird außerdem anhand eines Mikrochips der Unterschied zwischen Sekundärelektronenaufnahmen und Rückstreuelektronenaufnahmen deutlich gemacht.

  Die Transmissionselektronenmikroskopie wird genutzt, um verschiedene Teile eines Astrozyten aufzunehmen und die mikroskopische Strukturierung von Herzmuskeln ihrer Funktion zuzuordnen.
