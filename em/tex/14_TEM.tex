\section{TEM}

% ~5 Bilder
% what have we seen: charging artefacts (bad sample coating/high magnification, charge buildup)

%artefakte: kleine runde Wassereinschlüsse in der Plastikbeschichtung => Loch in der Probe
%schwarze streifen: irgendwelche Fehler in der Beschichtung
%schwarze Punkte: schwere Elemente (OsO4 die ja vorher für Kontrast da reingepackt wurden).
	%sO4 gibt Kontraste für Lipide.
	%r mehr Kontraste: Urandingens, etc. => differential staining, differential contrast
	%robe wurde nicht richtig abgewaschen => Kristalle dieser Salze als Kontamination auf der Oberfläche
%bild mit Gitter: kondensierte dna (nucleoli,) und Helferzelle, vat-grown (also zumindest nicht aus echtem Hirn)
%dann Helferzelle.
	%hne Kontrastmittel: auf reine Eiskristalle fokussieren, dann mit der Fokussierung Zytoplasma, Mitochondrien anschauen
	%resnel-(Ränder) um die Mitochondrien durch interferenz. Nucleus ist das große Runde.
%golgi(?)-Apparatus: Zysterny, einmal senkrecht, einmal parallel geschnischnitten
%im tem ist damage durch e-beam. runde helle fläche war in voriger vergrößerung fokussiert.
%Muskeln: quergestreifte Muskeln. Eine Zelle om menschl. Bizeps ist 15cm lang (Schulter bis ellenbogen), 1 Mikrometer abstand haben die schwarzen linien (Scheiben da dings)
	%ote Blutzelle oben links. bringt Glukose zu zelle, mitochondrien (Kreise) wandeln in ATP um=> Kontraktion der Mysin-Aktin-Dinger
	%chwarze Linien sind die Scheiben, wo das Aktin dranhängt und Myosin jeweils dazwischen
%irgendwas mit Platin: Platin in tiefen deposited (i think), "platinum replicas", sem-like image durch Invertierung
	%das Platin ist halt so als Form irgendwo die Topografie abgepauscht
